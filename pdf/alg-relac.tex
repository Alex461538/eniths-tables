% !TeX TS-program = lualatex
\documentclass{article}
\usepackage{pgfplots}
\usepackage{csquotes}
\usepackage{fontspec}
\usepackage[a4paper]{geometry}
\usepackage{babel}
\usepackage{multicol}
\usepackage{listings}
\usepackage{lstautogobble}
\usepackage{tikz}
\usepackage[most]{tcolorbox}
\usepackage[colorlinks=true, linkcolor=moonishblue, urlcolor=moonishblue]{hyperref}
\usepackage[dvipsnames]{xcolor}
\usepackage{emoji}
\usepackage{xcolor}
\usepackage{tabularx}

\definecolor{moonishblue}{RGB}{96, 30, 255}

\usetikzlibrary{arrows, positioning, shapes, fit, calc, angles, quotes}

\pgfdeclarelayer{background}
\pgfsetlayers{background,main}




\title{Una práctica ingeniosa para enseñar de todo. Más facil que tu hermana.}

\author{Los tres hijueputas del oriente}

% MapoGoldenPier

\newfontfamily\koreanfont[
Path = ./fonts/, % Adjust this if the folder is different
Extension = .ttf,
UprightFont = MapoGoldenPier
]{}

\newfontfamily\chinesefont[
Path = ./fonts/, % Adjust this if the folder is different
Extension = .ttf,
UprightFont = NotoSerifTC-SemiBold
]{}

\newcommand{\chin}[1]{{\chinesefont #1}}

\newcommand{\kor}[1]{{\koreanfont #1}}

\NewDocumentEnvironment{title-sign}{o}{%
	{\large \textbf{\href{https://www.youtube.com/@Pianitas38}{Alex} - \today}}%
}{}

\newenvironment{formula-box}{
	\begin{tcolorbox}[
		colframe=black, % color del borde
		colback=white, % color del fondo
		arc=1mm, % radio de las esquinas redondeadas
		boxrule=1pt, % grosor del borde
		left=2mm, % padding izquierdo
		right=2mm, % padding derecho
		top=2mm, % padding superior
		bottom=2mm, % padding inferior
		]
	}{
	\end{tcolorbox}
}

\newenvironment{title-box}[1]{
	\begin{tcolorbox}[
		colframe=white, % color del borde
		colback=white, % color del fondo
		arc=1mm, % radio de las esquinas redondeadas
		left=2mm, % padding izquierdo
		right=2mm, % padding derecho
		top=2mm, % padding superior
		bottom=2mm, % padding inferior
		borderline={0.5mm}{0mm}{#1,dashed},
		enhanced,
		boxrule=0.5mm,
		]
	}{
	\end{tcolorbox}
}

\newenvironment{cut-me-box}{
	\begin{tcolorbox}[
		colframe=white, % color del borde
		colback=white, % color del fondo
		arc=1mm, % radio de las esquinas redondeadas
		left=2mm, % padding izquierdo
		right=2mm, % padding derecho
		top=2mm, % padding superior
		bottom=2mm, % padding inferior
		borderline={0.5mm}{0mm}{black!70!white,dashed},
		enhanced,
		boxrule=0.5mm,
		overlay={
			\node[anchor=north, yshift=-1mm, fill=white] at (frame.north east) {\large \emoji{scissors}}; % Coloca el emoji sobre el borde superior
		}
		]
	}{
	\end{tcolorbox}
}

\lstset{
	xleftmargin=0pt,
	framexleftmargin=0pt,
	autogobble=true,
	linewidth=\linewidth,
	breaklines=true,
	columns=fullflexible,
	basicstyle=\small,
}

\begin{document}
	
	\newgeometry{left=0.7in,
		right=0.7in,
		top=0.9in,
		bottom=0.5in}
	
	{\centering \section*{DBs: Álgebra relacional}}
	
	Yo diría que el álgebra relacional se trata de una notación más o menos estandar y oscura para expresar consultas en bases de datos. Un ejemplo más de la amargura que nos traen los matemáticos.
	
	\setcounter{section}{0}
	
	\begin{multicols}{2}
		\section{Operadores relacionales?}
		
		\subsection{Selección}
		
		El equivalente perfecto de un SELECT, simplemente selecciona registros de un conjunto a partir de una condición.
		\[
		\sigma \text{ condición } ( datos )
		\]

		\subsection{Proyección}
		
		Es una variante de la selección, no tiene ninguna condición, pero te permite seleccionar los campos que desees de tus registros y elimina los duplicados resultantes.
		\[
		\pi \text{ campos,... } ( datos )
		\]
		
		\subsection{Renombre}
		
		Más orientada a la facilidad de manejo, se encarga de renombrar tablas o sus atributos.
		\[
		\begin{array}{c}
			\rho \text{ } nombre\_tabla( nombre\_campo,... ) (datos) \\
			\rho \text{ } nombre\_tabla (datos) \\
			\rho \text{ } campo \rightarrow nombre\_campo,... (datos)
		\end{array}
		\]
		
		\subsection{Ordenamiento}
		
		No muy diferente a un ORDER BY, simplemente ordena datos por una columna.
		\[
		\tau \text{ columna asc|desc } (datos)
		\]
		
		\subsection{Agrupación}
		
		Similar al GROUP BY, agrupa datos de acuerdo al valor de una columna y les aplica una función.
		\[
		\gamma \text{ } col,...; funcion(columna_2) \rightarrow col\_resultado (datos)
		\]
		Algunas de las funciones más comunes que se pueden utilizar aquí son:
		
		\begin{itemize}
			\item \textbf{count(columna):} Devuelve la cantidad de valores no nulos en la columna recibida, y en SQL puede juntarse con la palabra clave DISTINCT para ignorar duplicados.
			\item \textbf{sum(columna):} Calcula la suma de los valores de la columna recibida.
			\item \textbf{avg(columna):} Retorna el promedio de los datos en la columna recibida.
		\end{itemize}
		
		\section{Operadores de conjuntos}
		
		\subsection{Intersección}
		\subsection{Unión}
		\subsection{División}
		\subsection{Diferencia}
		
		\section{Joins}
		
		\subsection{Cross join}
		
		\subsection{Natural join}
		
		\subsection{Outer join}
		
		\subsubsection{Izquierdo (Left)}
		\subsubsection{Derecho (Right)}
		\subsubsection{Completo (Full)}
		
		\subsection{Semi join}
		
		\subsubsection{Izquierdo (Left)}
		\subsubsection{Derecho (Right)}
		
		\subsection{Anti join}
	
	\end{multicols}
	
\end{document}
