% !TeX TS-program = lualatex
\documentclass{article}
\usepackage{pgfplots}
\usepackage{csquotes}
\usepackage{fontspec}
\usepackage[a4paper]{geometry}
\usepackage{babel}
\usepackage{multicol}
\usepackage{listings}
\usepackage{lstautogobble}
\usepackage{tikz}
\usepackage[most]{tcolorbox}
\usepackage[colorlinks=true, linkcolor=moonishblue, urlcolor=moonishblue]{hyperref}
\usepackage[dvipsnames]{xcolor}
\usepackage{emoji}
\usepackage{xcolor}
\usepackage{tabularx}

\definecolor{moonishblue}{RGB}{96, 30, 255}

\usetikzlibrary{arrows, positioning, shapes, fit, calc, angles, quotes}

\pgfdeclarelayer{background}
\pgfsetlayers{background,main}




\title{Una práctica ingeniosa para enseñar de todo. Más facil que tu hermana.}

\author{Los tres hijueputas del oriente}

% MapoGoldenPier

\newfontfamily\koreanfont[
Path = ./fonts/, % Adjust this if the folder is different
Extension = .ttf,
UprightFont = MapoGoldenPier
]{}

\newfontfamily\chinesefont[
Path = ./fonts/, % Adjust this if the folder is different
Extension = .ttf,
UprightFont = NotoSerifTC-SemiBold
]{}

\newcommand{\chin}[1]{{\chinesefont #1}}

\newcommand{\kor}[1]{{\koreanfont #1}}

\NewDocumentEnvironment{title-sign}{o}{%
	{\large \textbf{\href{https://www.youtube.com/@Pianitas38}{Alex} - \today}}%
}{}

\newenvironment{formula-box}{
	\begin{tcolorbox}[
		colframe=black, % color del borde
		colback=white, % color del fondo
		arc=1mm, % radio de las esquinas redondeadas
		boxrule=1pt, % grosor del borde
		left=2mm, % padding izquierdo
		right=2mm, % padding derecho
		top=2mm, % padding superior
		bottom=2mm, % padding inferior
		]
	}{
	\end{tcolorbox}
}

\newenvironment{title-box}[1]{
	\begin{tcolorbox}[
		colframe=white, % color del borde
		colback=white, % color del fondo
		arc=1mm, % radio de las esquinas redondeadas
		left=2mm, % padding izquierdo
		right=2mm, % padding derecho
		top=2mm, % padding superior
		bottom=2mm, % padding inferior
		borderline={0.5mm}{0mm}{#1,dashed},
		enhanced,
		boxrule=0.5mm,
		]
	}{
	\end{tcolorbox}
}

\newenvironment{cut-me-box}{
	\begin{tcolorbox}[
		colframe=white, % color del borde
		colback=white, % color del fondo
		arc=1mm, % radio de las esquinas redondeadas
		left=2mm, % padding izquierdo
		right=2mm, % padding derecho
		top=2mm, % padding superior
		bottom=2mm, % padding inferior
		borderline={0.5mm}{0mm}{black!70!white,dashed},
		enhanced,
		boxrule=0.5mm,
		overlay={
			\node[anchor=north, yshift=-1mm, fill=white] at (frame.north east) {\large \emoji{scissors}}; % Coloca el emoji sobre el borde superior
		}
		]
	}{
	\end{tcolorbox}
}

\lstset{
	xleftmargin=0pt,
	framexleftmargin=0pt,
	autogobble=true,
	linewidth=\linewidth,
	breaklines=true,
	columns=fullflexible,
	basicstyle=\small,
}

\begin{document}
	
	\newgeometry{left=0.8in,
		right=0.8in,
		top=0.9in,
		bottom=0.5in}
	
	\begin{multicols}{2}
		\begin{title-box}{green!70!white}
			{\raggedright \large \textbf{Calculo I: Integrales irracionales} \par}
			
			\vspace{2mm}
			
			\begin{title-sign}\end{title-sign}
			
			\vspace{5mm}
			
			\emoji{four-leaf-clover} Una guía para integrar raíces. \emoji{four-leaf-clover}
		\end{title-box}
		
		\vfill
		
		\begin{formula-box}
			{\raggedright \large \textbf{Antes de comenzar} \par}
			
			Alguna vez nos habremos encontrado con integrales que tienen raíces de esta forma:
			\[
			\sqrt[n]{x^m}, \sqrt[n]{(ax+b)^m}, \sqrt[n]{( \frac{ax+b}{cx+d} )^m}
			\]
			Para términos de la forma no factorizable:
			\[
			\sqrt[n]{ax^2+bx+c}
			\]
			Se pueden completar cuadrados y aplicar la sustitución cuando sea lo más conveniente.
		\end{formula-box}
		
		\begin{formula-box}
			{\raggedright \large \textbf{Es muy simple} \par}
			
			Para esta sustitución se expresa el termino de la forma anterior como $t^k$, donde $k = MCM(n_1,n_2,...,n_i)$, por ejemplo:
			\[
			\begin{array}{c}
				\sqrt[\color{red} 3]{x^2}, \sqrt[\color{red} 2]{x} \Rightarrow x = t^6, k = MCM(3,2)
			\end{array}
			\]
			
			Seguido de esto se despeja $x$ con respecto a $t$ y se calcula $dx$ para realizar el cambio de variable. \par
			\vspace{2mm}
			Puede ser que la sustitución no elimine todos los radicales de una vez, en ese caso puede elegir entre:
			\begin{itemize}
				\item Repetir el proceso con los radicales faltantes.
				\item Sustitución trigonométrica.
				\item Completar cuadrados.
				\item Un método mejor para el caso especifico...
			\end{itemize}
		\end{formula-box}
		
		\columnbreak
		
		Bye
	\end{multicols}
	
\end{document}
