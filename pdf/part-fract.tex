% !TeX TS-program = lualatex
\documentclass{article}
\usepackage{pgfplots}
\usepackage{csquotes}
\usepackage{fontspec}
\usepackage[a4paper]{geometry}
\usepackage{babel}
\usepackage{multicol}
\usepackage{listings}
\usepackage{lstautogobble}
\usepackage{tikz}
\usepackage[most]{tcolorbox}
\usepackage[colorlinks=true, linkcolor=moonishblue, urlcolor=moonishblue]{hyperref}
\usepackage[dvipsnames]{xcolor}
\usepackage{emoji}
\usepackage{xcolor}
\usepackage{tabularx}

\definecolor{moonishblue}{RGB}{96, 30, 255}

\usetikzlibrary{arrows, positioning, shapes, fit, calc, angles, quotes}

\pgfdeclarelayer{background}
\pgfsetlayers{background,main}




\title{Una práctica ingeniosa para enseñar de todo. Más facil que tu hermana.}

\author{Los tres hijueputas del oriente}

% MapoGoldenPier

\newfontfamily\koreanfont[
Path = ./fonts/, % Adjust this if the folder is different
Extension = .ttf,
UprightFont = MapoGoldenPier
]{}

\newfontfamily\chinesefont[
Path = ./fonts/, % Adjust this if the folder is different
Extension = .ttf,
UprightFont = NotoSerifTC-SemiBold
]{}

\newcommand{\chin}[1]{{\chinesefont #1}}

\newcommand{\kor}[1]{{\koreanfont #1}}

\NewDocumentEnvironment{title-sign}{o}{%
	{\large \textbf{\href{https://www.youtube.com/@Pianitas38}{Alex} - \today}}%
}{}

\newenvironment{formula-box}{
	\begin{tcolorbox}[
		colframe=black, % color del borde
		colback=white, % color del fondo
		arc=1mm, % radio de las esquinas redondeadas
		boxrule=1pt, % grosor del borde
		left=2mm, % padding izquierdo
		right=2mm, % padding derecho
		top=2mm, % padding superior
		bottom=2mm, % padding inferior
		]
	}{
	\end{tcolorbox}
}

\newenvironment{title-box}[1]{
	\begin{tcolorbox}[
		colframe=white, % color del borde
		colback=white, % color del fondo
		arc=1mm, % radio de las esquinas redondeadas
		left=2mm, % padding izquierdo
		right=2mm, % padding derecho
		top=2mm, % padding superior
		bottom=2mm, % padding inferior
		borderline={0.5mm}{0mm}{#1,dashed},
		enhanced,
		boxrule=0.5mm,
		]
	}{
	\end{tcolorbox}
}

\newenvironment{cut-me-box}{
	\begin{tcolorbox}[
		colframe=white, % color del borde
		colback=white, % color del fondo
		arc=1mm, % radio de las esquinas redondeadas
		left=2mm, % padding izquierdo
		right=2mm, % padding derecho
		top=2mm, % padding superior
		bottom=2mm, % padding inferior
		borderline={0.5mm}{0mm}{black!70!white,dashed},
		enhanced,
		boxrule=0.5mm,
		overlay={
			\node[anchor=north, yshift=-1mm, fill=white] at (frame.north east) {\large \emoji{scissors}}; % Coloca el emoji sobre el borde superior
		}
		]
	}{
	\end{tcolorbox}
}

\lstset{
	xleftmargin=0pt,
	framexleftmargin=0pt,
	autogobble=true,
	linewidth=\linewidth,
	breaklines=true,
	columns=fullflexible,
	basicstyle=\small,
}

\begin{document}
	
	\newgeometry{left=0.8in,
		right=0.8in,
		top=0.9in,
		bottom=0.5in}
	
	\begin{multicols}{2}
		\begin{title-box}{moonishblue!70!white}
			{\raggedright \large \textbf{Calculo I: Fracciones parciales} \par}
			
			\vspace{2mm}
			
			\begin{title-sign}\end{title-sign}
			
			\vspace{5mm}
			
			\emoji{glowing-star} Una guía para dividir fracciones. \emoji{glowing-star}
		\end{title-box}
		
		\vfill
		
		\begin{formula-box}
			{\raggedright \large \textbf{Antes de comenzar} \par}
			
			En algunos casos al integrar encontraremos fracciones como esta:
			\[
			\frac{P_n(x)}{Q_m(x)}
			\]
			
			En estos casos una posible solución podría ser separar la fracción en una suma de fracciones parciales.
		\end{formula-box}
		
		\begin{formula-box}
			{\raggedright \large \textbf{Fracciones impropias} \par}
			
			Una fracción se llama impropia cuando $n \geq m$, en este caso se dividen los polinomios:
			\[
			\frac{P_n(x)}{Q_m(x)} = C(x) + \frac{R(x)}{Q(x)}
			\]
			
			Donde $C(x)$ es el cociente de la división y $R(x)$ el residuo
		\end{formula-box}
		
		\begin{formula-box}
			{\raggedright \large \textbf{Fracciones propias} \par}
			
			Una fracción se llama propia cuando $n < m$. En este caso se debe comprobar si $Q_m(x)$ es un polinomio factorizable en términos de la forma $(x-a)^k, (x^2+bx+c)^k, k \geq 1$, si lo anterior se cumple se realiza una descomposición en fracciones parciales.
		\end{formula-box}
		
		\columnbreak
		
		\begin{formula-box}
			{\raggedright \large \textbf{Descomposición en fracciones parciales} \par}
			
			Consiste en reescribir una fracción como una suma finita de fracciones mas simples. Dependiendo se los factores de $Q(x)$ se escribe la suma así para cada uno:
			
			\vspace{2mm}
			
			\begin{tabularx}{\linewidth}{|p{3cm}|X|}
				\hline
				$(x-a)^k$ & $ \frac{A_1}{x-a} + \frac{A_2}{(x-a)^2} + \dots + \frac{A_k}{(x-a)^k} $ \\
				\hline
				$(x^2+bx+c)^k$ & $ \frac{A_1x+B_1}{x^2+bx+c} + \frac{A_2x+B_2}{(x^2+bx+c)^2} + \dots + \frac{A_kx+B_k}{(x^2+bx+c)^k} $ \\
				\hline
			\end{tabularx}
			
			\vspace{2mm}
			
			\textbf{Tenga en cuanta que si hay multiples tipos de factores se escribe una combinación de los casos}
			
			Para hallar las constantes se iguala la suma con la fracción original, por ejemplo:
			
			\[
			\frac{1}{x^2(x^2+1)} = \frac{A_1}{x} + \frac{A_2}{x^2} + \frac{A_3x+B_1}{x^2+1}
			\]
			
			Luego se multiplica para quitar los denominadores y se factoriza la expresión con respecto a $x$:
			\[
			\begin{array}{c}
				1 = A_1x(x^2+1) + A_2(x^2+1) +(A_3x+B_1)x^2 \\
				= x^3(A_1+A_3) + x^2(A_2+B_1) + A_1x + A_2
			\end{array}
			\]
			Al final se obtiene a partir de la igualdad de polinomios un sistema de ecuaciones para despejar las constantes.
			\[
			\begin{array}{c|c}
				\begin{cases}
					A_1+A_3 = 0 \\
					A_2+B_1 = 0 \\
					A_1 = 0 \\
					A_2 = 1
				\end{cases} & \begin{cases}
					A_1 = 0 \\
					A_2 = 1 \\
					A_3 = 0 \\
					B_1 = -1
				\end{cases}
			\end{array}
			\]
			
			Por ultimo se reemplazan las constantes para revelar el resultado:
			\[
			\frac{1}{x^2(x^2+1)} = \frac{1}{x^2} - \frac{1}{x^2+1}
			\]
		\end{formula-box}
	\end{multicols}
	
\end{document}
