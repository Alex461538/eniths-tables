% !TeX TS-program = lualatex
\documentclass{article}
\usepackage{pgfplots}
\usepackage{csquotes}
\usepackage{fontspec}
\usepackage[a4paper]{geometry}
\usepackage{babel}
\usepackage{multicol}
\usepackage{listings}
\usepackage{lstautogobble}
\usepackage{tikz}
\usepackage[most]{tcolorbox}
\usepackage[colorlinks=true, linkcolor=moonishblue, urlcolor=moonishblue]{hyperref}
\usepackage[dvipsnames]{xcolor}
\usepackage{emoji}
\usepackage{xcolor}
\usepackage{tabularx}

\definecolor{moonishblue}{RGB}{96, 30, 255}

\usetikzlibrary{arrows, positioning, shapes, fit, calc, angles, quotes}

\pgfdeclarelayer{background}
\pgfsetlayers{background,main}




\title{Una práctica ingeniosa para enseñar de todo. Más facil que tu hermana.}

\author{Los tres hijueputas del oriente}

% MapoGoldenPier

\newfontfamily\koreanfont[
Path = ./fonts/, % Adjust this if the folder is different
Extension = .ttf,
UprightFont = MapoGoldenPier
]{}

\newfontfamily\chinesefont[
Path = ./fonts/, % Adjust this if the folder is different
Extension = .ttf,
UprightFont = NotoSerifTC-SemiBold
]{}

\newcommand{\chin}[1]{{\chinesefont #1}}

\newcommand{\kor}[1]{{\koreanfont #1}}

\NewDocumentEnvironment{title-sign}{o}{%
	{\large \textbf{\href{https://www.youtube.com/@Pianitas38}{Alex} - \today}}%
}{}

\newenvironment{formula-box}{
	\begin{tcolorbox}[
		colframe=black, % color del borde
		colback=white, % color del fondo
		arc=1mm, % radio de las esquinas redondeadas
		boxrule=1pt, % grosor del borde
		left=2mm, % padding izquierdo
		right=2mm, % padding derecho
		top=2mm, % padding superior
		bottom=2mm, % padding inferior
		]
	}{
	\end{tcolorbox}
}

\newenvironment{title-box}[1]{
	\begin{tcolorbox}[
		colframe=white, % color del borde
		colback=white, % color del fondo
		arc=1mm, % radio de las esquinas redondeadas
		left=2mm, % padding izquierdo
		right=2mm, % padding derecho
		top=2mm, % padding superior
		bottom=2mm, % padding inferior
		borderline={0.5mm}{0mm}{#1,dashed},
		enhanced,
		boxrule=0.5mm,
		]
	}{
	\end{tcolorbox}
}

\newenvironment{cut-me-box}{
	\begin{tcolorbox}[
		colframe=white, % color del borde
		colback=white, % color del fondo
		arc=1mm, % radio de las esquinas redondeadas
		left=2mm, % padding izquierdo
		right=2mm, % padding derecho
		top=2mm, % padding superior
		bottom=2mm, % padding inferior
		borderline={0.5mm}{0mm}{black!70!white,dashed},
		enhanced,
		boxrule=0.5mm,
		overlay={
			\node[anchor=north, yshift=-1mm, fill=white] at (frame.north east) {\large \emoji{scissors}}; % Coloca el emoji sobre el borde superior
		}
		]
	}{
	\end{tcolorbox}
}

\lstset{
	xleftmargin=0pt,
	framexleftmargin=0pt,
	autogobble=true,
	linewidth=\linewidth,
	breaklines=true,
	columns=fullflexible,
	basicstyle=\small,
}

\begin{document}
	
	\newgeometry{left=0.8in,
		right=0.8in,
		top=0.9in,
		bottom=0.5in}
	
	{\centering \section*{Arquitectura de computadores: caché}}
	
	La memoria caché permite aumentar la velocidad de nuestros programas, pues la memoria RAM es muy lenta. En esencia, la caché es una memoria intermedia, más cercana a la CPU, que permite guardar temporalmente porciones de la memoria principal (bloques) mientras son usadas y así disminuir los accesos a la memoria principal.
	
	\setcounter{section}{0}
	
	\begin{multicols}{2}
		\section{Mapeo de cache}
		
		Se refiere a las formas que existen de interpretar las direcciones en memoria para situar sus bloques en la cache.
		
		En una cache de N lineas y un tamaño de bloque S donde N y S son potencias de 2 se tienen los siguientes tipos de mapeo:
		
		\subsection{Mapeo directo}
		
		\begin{center}
			\begin{tabular}{|c|c|c|c|}
				\hline
				name & tag & line & word \\
				\hline
				example & FF & 6543 & 21 \\
				\hline
				size & remaining & $\log_2N$ & $\log_2S$ \\
				\hline
			\end{tabular}
		\end{center}
		
		En este mapeo, \textbf{Word} define la posición de un dato dentro del bloque. \textbf{Line} la linea de cache fija donde la dirección de memoria debería situarse, y \textbf{Tag} sirve como un diferenciador. Este mapeo no permite un mejor uso de la cache con políticas de reemplazo.
		
		\subsection{Mapeo totalmente asociativo}
		
		\begin{center}
			\begin{tabular}{|c|c|c|}
				\hline
				name & tag & word \\
				\hline
				example & FF6543 & 21 \\
				\hline
				size & remaining & $\log_2S$ \\
				\hline
			\end{tabular}
		\end{center}
		
		En este mapeo solo encontramos a \textbf{Word} y \textbf{Tag}, lo cual permite una asignación dinámica de lineas con políticas de reemplazo, aunque es muy costoso de fabricar.
		
		\subsection{Mapeo asociativo por conjuntos}
		
		Considere a E como el número de conjuntos donde E es potencia de 2.
		
		\begin{center}
			\begin{tabular}{|c|c|c|c|}
				\hline
				name & tag & set & word \\
				\hline
				example & FF654 & 3 & 21 \\
				\hline
				size & remaining & $\log_2\frac{N}{E}$ & $\log_2S$ \\
				\hline
			\end{tabular}
		\end{center}
		
		Este mapeo ofrece un balance entre las alternativas anteriores. Se particiona la cache en E conjuntos. Cada bloque de memoria puede pertenecer solo a un conjunto fijo, pero la posición dentro del mismo es dinámica con la \textbf{Tag}. El numero de lineas por conjunto $\frac{N}{E}$ denota las \textbf{Vías} asociativas.
		
		{\footnotesize Más información sobre mapeos en \href{https://www.geeksforgeeks.org/computer-organization-architecture/difference-between-direct-mapping-associative-mapping-set-associative-mapping/}{Geeksforgeeks}}
		
		\section{Políticas de reemplazo}
		
		\subsection{FIFO o Round Robin}
		
		Asigna lineas de cache con un contador circular (Wraps on overflow). Es fácil de implementar, pero no tiene en cuanta el uso de cada linea.
		
		\subsection{LRU o Least Recently Used}
		
		Descarta lineas de cache basándose en una prioridad que denota qué tan usadas fueron. Es eficiente, aunque necesita muchos bits de control y por ende es muy costosa.
		
		\subsection{Aleatoria}
		
		Su efectividad depende de la suerte, aunque entre más lineas de cache, es menos probable descartar las más importantes.
		
		\section{Políticas de escritura}
		
		\subsection{[Hit] Write through}
		
		Escribe en la cache y memoria principal en paralelo. Esto aumenta el tráfico en el BUS de datos.
		
		\subsection{[Hit] Write back}
		
		Se escribe todo en la cache y solo se actualizan los datos en memoria principal cuando el bloque va a ser descartado. Requiere un bit de control \textbf{dirty}.
		
		\subsection{[Miss] Write allocate}
		
		Trae a la cache el bloque solicitado y escribe.
		
		\subsection{[Miss] No write allocate}
		
		Se escribe en memoria principal sin traer el bloque a cache.
		
		\section{Políticas de captación}
		
		\subsection{Demand}
		
		Cuando hay un miss en la lectura, trae el bloque solicitado a la cache.
		
		\subsection{Prefetch Always}
		
		Cuando haya una lectura traerá los bloques cercanos en hits y fallos.
		
		\subsection{Prefetch miss}
		
		Solo trae los bloques cercanos en fallos
	\end{multicols}
	
\end{document}
