% !TeX TS-program = lualatex
\documentclass{article}
\usepackage{pgfplots}
\usepackage{csquotes}
\usepackage{fontspec}
\usepackage[a4paper]{geometry}
\usepackage{babel}
\usepackage{multicol}
\usepackage{listings}
\usepackage{enumitem}
\usepackage{lstautogobble}
\usepackage{tikz}
\usetikzlibrary{positioning,calc}
\usepackage[most]{tcolorbox}
\usepackage[colorlinks=true, linkcolor=moonishblue, urlcolor=moonishblue]{hyperref}
\usepackage[dvipsnames]{xcolor}
\usepackage{emoji}
\usepackage{xcolor}
\usepackage{tabularx}

\definecolor{moonishblue}{RGB}{96, 30, 255}

\usetikzlibrary{arrows, positioning, shapes, fit, calc, angles, quotes}

\pgfdeclarelayer{background}
\pgfsetlayers{background,main}




\title{Una práctica ingeniosa para enseñar de todo. Más facil que tu hermana.}

\author{Los tres hijueputas del oriente}

% MapoGoldenPier

\newfontfamily\koreanfont[
Path = ./fonts/, % Adjust this if the folder is different
Extension = .ttf,
UprightFont = MapoGoldenPier
]{}

\newfontfamily\chinesefont[
Path = ./fonts/, % Adjust this if the folder is different
Extension = .ttf,
UprightFont = NotoSerifTC-SemiBold
]{}

\newcommand{\chin}[1]{{\chinesefont #1}}

\newcommand{\kor}[1]{{\koreanfont #1}}

\NewDocumentEnvironment{title-sign}{o}{%
	{\large \textbf{\href{https://www.youtube.com/@Pianitas38}{Alex} - \today}}%
}{}

\newenvironment{formula-box}{
	\begin{tcolorbox}[
		colframe=black, % color del borde
		colback=white, % color del fondo
		arc=1mm, % radio de las esquinas redondeadas
		boxrule=1pt, % grosor del borde
		left=2mm, % padding izquierdo
		right=2mm, % padding derecho
		top=2mm, % padding superior
		bottom=2mm, % padding inferior
		]
	}{
	\end{tcolorbox}
}

\newenvironment{title-box}[1]{
	\begin{tcolorbox}[
		colframe=white, % color del borde
		colback=white, % color del fondo
		arc=1mm, % radio de las esquinas redondeadas
		left=2mm, % padding izquierdo
		right=2mm, % padding derecho
		top=2mm, % padding superior
		bottom=2mm, % padding inferior
		borderline={0.5mm}{0mm}{#1,dashed},
		enhanced,
		boxrule=0.5mm,
		]
	}{
	\end{tcolorbox}
}

\newenvironment{cut-me-box}{
	\begin{tcolorbox}[
		colframe=white, % color del borde
		colback=white, % color del fondo
		arc=1mm, % radio de las esquinas redondeadas
		left=2mm, % padding izquierdo
		right=2mm, % padding derecho
		top=2mm, % padding superior
		bottom=2mm, % padding inferior
		borderline={0.5mm}{0mm}{black!70!white,dashed},
		enhanced,
		boxrule=0.5mm,
		overlay={
			\node[anchor=north, yshift=-1mm, fill=white] at (frame.north east) {\large \emoji{scissors}}; % Coloca el emoji sobre el borde superior
		}
		]
	}{
	\end{tcolorbox}
}

\lstset{
	xleftmargin=0pt,
	framexleftmargin=0pt,
	autogobble=true,
	linewidth=\linewidth,
	breaklines=true,
	columns=fullflexible,
	basicstyle=\small,
}

\begin{document}
	
	\newgeometry{left=0.8in,
		right=0.8in,
		top=0.9in,
		bottom=0.9in}
	
	{\centering \section*{Parseo de expresiones con arboles (Ella no aprende)}}
	
	Si alguna vez han rechazado tu algoritmo de parseo de expresiones por no tener pila... ¡no te preocupes! Un día se me ocurrió en paint este algoritmo que usa árboles y es muy eficiente. Si ya existía seguro fue olvidado, seguro el autor está muerto, no es importante.
	
	\setcounter{section}{0}
	
	\begin{multicols}{2}
		\section{Antes de comenzar...}
		
		Primero considera que usaremos un arbol binario tradicional, donde el valor de cada nodo es de tipo token.
		Además, tienes una lista de tokens: operadores binarios con una prioridad, y simbolos.
		
		\[
		\begin{array}{c}
			tokens = [\{"name": "1", "type":TK\_ID, "prior":0\}, \\
			\{"name": "+", "type":TK\_OPR, "prior":1\} ...]
		\end{array}
		\]
		
		Luego añadir unarios, errores y parentesis es notablemente natural.
		
		\section{Pseudocódigo}
		
		La eficiencia de esto es casi lineal, con el detalle del bucle, en el peor de los casos sospecho que esto da un poco menos de $P * n$ pasos, donde $P$ son todas las prioridades diferentes que se manejan.
		
		\begin{lstlisting}[language=python]
			tree = tokens[0]
			root = tree
			for token in tokens[1...n]:
				# Empuja abajo a la izquierda
				if token is operator:
					next = Node(token)
					next.left = root
					root = next
				else if token is operand and not root.right:
					# Asigna el hijo derecho
					root.right = token
					aux = root
					while aux and aux.left are operators and aux.left.priority > aux.priority:
						# Rota hacia la derecha
						old = aux
						aux = aux.left
						old.left = aux.right
						aux.right = old
						# Profundiza
						aux = aux.right
		\end{lstlisting}
		
		\section{A practicar}
		
		Supongamos que tenemos estos tokens: $a,*,b,+,c,*,d$, y las prioridades de $+$ y $*$ son $1$ y $0$ respectivamente:
		
		\begin{center}
			[1]: Añade el primer token.
			
			\begin{tikzpicture}[node distance=2cm, every node/.style={circle, draw, minimum size=7mm}]
				\node (1) {a};
			\end{tikzpicture}
			
			[2]: Añade un operador, empuja abajo izquierda.
			
			\begin{tikzpicture}[node distance=2cm, every node/.style={circle, draw, minimum size=7mm}]
				\node (L) {*};
				\node (1) [below left=0.5cm of L] {a};
				
				\draw [->] (L) -- (1);
			\end{tikzpicture}
			
			\columnbreak
			
			[3]: Añade un operando, no hay rotación.
			
			\begin{tikzpicture}[node distance=2cm, every node/.style={circle, draw, minimum size=7mm}]
				\node (L) {*};
				\node (1) [below left=0.5cm of L] {a};
				\node (2) [below right=0.5cm of L] {b};
				
				\draw [->] (L) -- (1);
				\draw [->] (L) -- (2);
			\end{tikzpicture}
			
			\vspace{1mm}
			
			[4]: Añade un operador, empuja abajo izquierda.
			
			\begin{tikzpicture}[node distance=2cm, every node/.style={circle, draw, minimum size=7mm}]
				\node (M) {+};
				
				\node (L) [below left=0.5cm of M] {*};
				\node (1) [below left=0.5cm of L] {a};
				\node (2) [below right=0.5cm of L] {b};
				
				\draw [->] (M) -- (L);
				
				\draw [->] (L) -- (1);
				\draw [->] (L) -- (2);
			\end{tikzpicture}
			
			\vspace{1mm}
			
			[5]: Añade un operando, no hay rotación.
			
			\begin{tikzpicture}[node distance=2cm, every node/.style={circle, draw, minimum size=7mm}]
				\node (M) {+};
				
				\node (L) [below left=0.5cm of M] {*};
				\node (1) [below left=0.5cm of L] {a};
				\node (2) [below right=0.5cm of L] {b};
				
				\node (3) [below right=0.5cm of M] {c};
				
				\draw [->] (M) -- (L);
				\draw [->] (M) -- (3);
				
				\draw [->] (L) -- (1);
				\draw [->] (L) -- (2);
			\end{tikzpicture}
			
			\vspace{1mm}
			
			[6]: Añade un operador, empuja abajo izquierda.
			
			\begin{tikzpicture}[node distance=2cm, every node/.style={circle, draw, minimum size=7mm}]
				\node (S) {*};
				
				\node (M) [below left=0.5cm of S] {+};
				
				\node (L) [below left=0.5cm of M] {*};
				\node (1) [below left=0.5cm of L] {a};
				\node (2) [below right=0.5cm of L] {b};
				
				\node (3) [below right=0.5cm of M] {c};
				
				\draw [->] (S) -- (M);
				
				\draw [->] (M) -- (L);
				\draw [->] (M) -- (3);
				
				\draw [->] (L) -- (1);
				\draw [->] (L) -- (2);
			\end{tikzpicture}
			
			\vspace{1mm}
			
			[7]: Añade un operando, hay una rotación.
			
			\begin{tikzpicture}[node distance=2cm, every node/.style={circle, draw, minimum size=7mm}]
				\node (S) {*};
				
				\node (M) [below left=0.5cm of S] {+};
				
				\node (L) [below left=0.5cm of M] {*};
				\node (1) [below left=0.5cm of L] {a};
				\node (2) [below right=0.5cm of L] {b};
				
				\node (3) [below right=0.5cm of M] {c};
				\node (4) [below right=0.5cm of S] {d};
				
				\draw [->] (S) -- (M);
				
				\draw [->] (M) -- (L);
				\draw [->] (M) -- (3);
				
				\draw [->] (L) -- (1);
				\draw [->] (L) -- (2);
				
				\draw [->] (S) -- (4);
				
				\draw[->, dashed] (M) .. controls +(up:1.7cm) and +(up:1.7cm) .. (4);
				
			\end{tikzpicture}
			
			\vspace{1mm}
			
			[8] No hay más rotación, no hay más tokens, termina.
			
			\begin{tikzpicture}[node distance=2cm, every node/.style={circle, draw, minimum size=7mm}]
				\node (M) {+};
				
				\node (L) [below left of=M] {*};
				\node (1) [below left=0.5cm of L] {a};
				\node (2) [below right=0.5cm of L] {b};
				
				\node (R) [below right of=M] {*};
				\node (3) [below left=0.5cm of R] {c};
				\node (4) [below right=0.5cm of R] {d};
				
				\draw [->] (M) -- (L);
				\draw [->] (M) -- (R);
				
				\draw [->] (L) -- (1);
				\draw [->] (L) -- (2);
				
				\draw [->] (R) -- (3);
				\draw [->] (R) -- (4);
			\end{tikzpicture}
		\end{center}
		
		
	\end{multicols}
\end{document}
