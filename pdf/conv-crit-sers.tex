% !TeX TS-program = lualatex
\documentclass{article}
\usepackage{pgfplots}
\usepackage{csquotes}
\usepackage{fontspec}
\usepackage[a4paper]{geometry}
\usepackage{babel}
\usepackage{amsmath}
\usepackage{amssymb}
\usepackage{multicol}
\usepackage{listings}
\usepackage{lstautogobble}
\usepackage{tikz}
\usepackage[most]{tcolorbox}
\usepackage[colorlinks=true, linkcolor=moonishblue, urlcolor=moonishblue]{hyperref}
\usepackage[dvipsnames]{xcolor}
\usepackage{emoji}
\usepackage{xcolor}
\usepackage{tabularx}

\definecolor{moonishblue}{RGB}{96, 30, 255}

\usetikzlibrary{arrows, positioning, shapes, fit, calc, angles, quotes}

\pgfdeclarelayer{background}
\pgfsetlayers{background,main}




\title{Una práctica ingeniosa para enseñar de todo. Más facil que tu hermana.}

\author{Los tres hijueputas del oriente}

% MapoGoldenPier

\newfontfamily\koreanfont[
Path = ./fonts/, % Adjust this if the folder is different
Extension = .ttf,
UprightFont = MapoGoldenPier
]{}

\newfontfamily\chinesefont[
Path = ./fonts/, % Adjust this if the folder is different
Extension = .ttf,
UprightFont = NotoSerifTC-SemiBold
]{}

\newcommand{\chin}[1]{{\chinesefont #1}}

\newcommand{\kor}[1]{{\koreanfont #1}}

\NewDocumentEnvironment{title-sign}{o}{%
	{\large \textbf{\href{https://www.youtube.com/@Pianitas38}{Alex} - \today}}%
}{}

\newenvironment{formula-box}{
	\begin{tcolorbox}[
		colframe=black, % color del borde
		colback=white, % color del fondo
		arc=1mm, % radio de las esquinas redondeadas
		boxrule=1pt, % grosor del borde
		left=2mm, % padding izquierdo
		right=2mm, % padding derecho
		top=2mm, % padding superior
		bottom=2mm, % padding inferior
		]
	}{
	\end{tcolorbox}
}

\newenvironment{title-box}[1]{
	\begin{tcolorbox}[
		colframe=white, % color del borde
		colback=white, % color del fondo
		arc=1mm, % radio de las esquinas redondeadas
		left=2mm, % padding izquierdo
		right=2mm, % padding derecho
		top=2mm, % padding superior
		bottom=2mm, % padding inferior
		borderline={0.5mm}{0mm}{#1,dashed},
		enhanced,
		boxrule=0.5mm,
		]
	}{
	\end{tcolorbox}
}

\newenvironment{cut-me-box}{
	\begin{tcolorbox}[
		colframe=white, % color del borde
		colback=white, % color del fondo
		arc=1mm, % radio de las esquinas redondeadas
		left=2mm, % padding izquierdo
		right=2mm, % padding derecho
		top=2mm, % padding superior
		bottom=2mm, % padding inferior
		borderline={0.5mm}{0mm}{black!70!white,dashed},
		enhanced,
		boxrule=0.5mm,
		overlay={
			\node[anchor=north, yshift=-1mm, fill=white] at (frame.north east) {\large \emoji{scissors}}; % Coloca el emoji sobre el borde superior
		}
		]
	}{
	\end{tcolorbox}
}

\lstset{
	xleftmargin=0pt,
	framexleftmargin=0pt,
	autogobble=true,
	linewidth=\linewidth,
	breaklines=true,
	columns=fullflexible,
	basicstyle=\small,
}

\begin{document}
	
	\newgeometry{left=0.8in,
		right=0.8in,
		top=0.9in,
		bottom=0.5in}
	
	\begin{multicols}{2}
		\begin{title-box}{green!70!white}
			{\raggedright \large \textbf{Series y sucesiones: Criterios de convergencia} \par}
			
			\vspace{2mm}
			
			\begin{title-sign}\end{title-sign}
			
			\vspace{5mm}
			
			\emoji{four-leaf-clover} Y la que converja. \emoji{four-leaf-clover}
		\end{title-box}
		
		\vfill
		
		\begin{formula-box}
			{\raggedright \large \textbf{Antes de comenzar} \par}
			
			Estas técnicas pueden ser usadas cuando solo se quiere saber si una serie converge o diverge y no el resultado específico.
		\end{formula-box}
		
		\begin{formula-box}
			{\raggedright \large \textbf{Criterio de la razón} \par}
			
			El criterio consiste en que si tienes una sucesión puedes plantear este límite:
			
			\[
			\lim_{n \rightarrow \infty } \frac{a_{n+1}}{a_n} = r
			\]
			
			Si $r > 1$ La serie diverge, si $r < 1$ converge, y si $r = 1$ el criterio no es concluyente.
		\end{formula-box}
		
		\begin{formula-box}
			{\raggedright \large \textbf{Criterio de la integral} \par}
			
			Si se tiene una sucesión y sacamos su termino $a_n$ en una función $a_n = f(x)$, siempre que $f(x)$ sea continua, positiva y decreciente en $[1, \infty]$ podemos decir que:
			
			\[
			\sum_{n=1}^{\infty} a_n \text{ converge} \iff \int_{1}^{\infty} f(x) \text{ converge}
			\]
		\end{formula-box}
		
		\begin{formula-box}
			{\raggedright \large \textbf{Criterio de la raiz} \par}
			
			content...
		\end{formula-box}
		
		\begin{formula-box}
			{\raggedright \large \textbf{Criterio de comparación} \par}
			
			Si tengo dos series $\sum_{n=1}^{\infty} a_n$ y $\sum_{n=1}^{\infty} b_n$ con terminos positivos, este criterio dicta que:
			
			\begin{itemize}
				\item Si $\sum_{n=1}^{\infty} b_n$ converge y $a_n < b_n$ para $n \in \mathbb{Z}^+$ entonces $\sum_{n=1}^{\infty} a_n$ converge.
				
				\item Si $\sum_{n=1}^{\infty} b_n$ diverge y $a_n > b_n$ para $n \in \mathbb{Z}^+$ entonces $\sum_{n=1}^{\infty} a_n$ diverge.
			\end{itemize}
			
		\end{formula-box}
		
		\columnbreak
		
		\begin{cut-me-box}
			{\raggedright \large \textbf{Ahora a practicar} \par}
			
			\[
			\sum_{n=1}^{\infty} 4(\frac{2}{3})^n
			\]
			
			Sacamos constante y trabajamos facilmente:
			
			\[
			\sum_{n=1}^{\infty} 4(\frac{2}{3})^n =
			4 \sum_{n=1}^{\infty} (\frac{2}{3})^n
			\]
			
			Luego cambiamos los índices y ajustamos para trabajar con la fórmula anterior:
			
			\[
			4 \sum_{n=1}^{\infty} (\frac{2}{3})^n = 4 (\sum_{n=0}^{\infty} (\frac{2}{3})^n - (\frac{2}{3})^0)
			\]
			
			Y ahora sí, reemplazamos nuestra fórmula general:
			
			\[
			\begin{array}{c}
				c = 4( \frac{1}{1-r} - 1) \\
				c = 4( \frac{1}{1-2/3} - 1) \\
				c = 4( 3 - 1) = 8
			\end{array}
			\]
			
			\[
			\sum_{n=1}^{\infty} ( 2^{-n} + 3^{-n} )
			\]
			
			Parece dificil, pero podemos trabajarla para que mejore:
			
			\[
			y = \sum_{n=1}^{\infty} ( 2^{-n} + 3^{-n} ) = \sum_{n=1}^{\infty} (\frac{1}{2})^n + \sum_{n=1}^{\infty} (\frac{1}{3})^n
			\]
			
			Y seguido de esto cambiamos los índices:
			
			\[
			\frac{1}{2} \sum_{n=0}^{\infty} (\frac{1}{2})^n + \frac{1}{3} \sum_{n=0}^{\infty} (\frac{1}{3})^n
			\]
			
			Y ya queda esto bien pelado:
			
			\[
			\begin{array}{c}
				c = \frac{1}{2}( \frac{1}{1-1/2} ) + \frac{1}{3}( \frac{1}{1 - 1/3} ) \\
				c = 1 + 1/2 = 3/2
			\end{array}
			\]
			
		\end{cut-me-box}
	\end{multicols}
	
\end{document}
