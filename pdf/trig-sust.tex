% !TeX TS-program = lualatex
\documentclass{article}
\usepackage{pgfplots}
\usepackage{csquotes}
\usepackage{fontspec}
\usepackage[a4paper]{geometry}
\usepackage{babel}
\usepackage{multicol}
\usepackage{listings}
\usepackage{lstautogobble}
\usepackage{tikz}
\usepackage[most]{tcolorbox}
\usepackage[colorlinks=true, linkcolor=moonishblue, urlcolor=moonishblue]{hyperref}
\usepackage[dvipsnames]{xcolor}
\usepackage{emoji}
\usepackage{xcolor}
\usepackage{tabularx}

\definecolor{moonishblue}{RGB}{96, 30, 255}

\usetikzlibrary{arrows, positioning, shapes, fit, calc, angles, quotes}

\pgfdeclarelayer{background}
\pgfsetlayers{background,main}




\title{Una práctica ingeniosa para enseñar de todo. Más facil que tu hermana.}

\author{Los tres hijueputas del oriente}

% MapoGoldenPier

\newfontfamily\koreanfont[
Path = ./fonts/, % Adjust this if the folder is different
Extension = .ttf,
UprightFont = MapoGoldenPier
]{}

\newfontfamily\chinesefont[
Path = ./fonts/, % Adjust this if the folder is different
Extension = .ttf,
UprightFont = NotoSerifTC-SemiBold
]{}

\newcommand{\chin}[1]{{\chinesefont #1}}

\newcommand{\kor}[1]{{\koreanfont #1}}

\NewDocumentEnvironment{title-sign}{o}{%
	{\large \textbf{\href{https://www.youtube.com/@Pianitas38}{Alex} - \today}}%
}{}

\newenvironment{formula-box}{
	\begin{tcolorbox}[
		colframe=black, % color del borde
		colback=white, % color del fondo
		arc=1mm, % radio de las esquinas redondeadas
		boxrule=1pt, % grosor del borde
		left=2mm, % padding izquierdo
		right=2mm, % padding derecho
		top=2mm, % padding superior
		bottom=2mm, % padding inferior
		]
	}{
	\end{tcolorbox}
}

\newenvironment{title-box}[1]{
	\begin{tcolorbox}[
		colframe=white, % color del borde
		colback=white, % color del fondo
		arc=1mm, % radio de las esquinas redondeadas
		left=2mm, % padding izquierdo
		right=2mm, % padding derecho
		top=2mm, % padding superior
		bottom=2mm, % padding inferior
		borderline={0.5mm}{0mm}{#1,dashed},
		enhanced,
		boxrule=0.5mm,
		]
	}{
	\end{tcolorbox}
}

\newenvironment{cut-me-box}{
	\begin{tcolorbox}[
		colframe=white, % color del borde
		colback=white, % color del fondo
		arc=1mm, % radio de las esquinas redondeadas
		left=2mm, % padding izquierdo
		right=2mm, % padding derecho
		top=2mm, % padding superior
		bottom=2mm, % padding inferior
		borderline={0.5mm}{0mm}{black!70!white,dashed},
		enhanced,
		boxrule=0.5mm,
		overlay={
			\node[anchor=north, yshift=-1mm, fill=white] at (frame.north east) {\large \emoji{scissors}}; % Coloca el emoji sobre el borde superior
		}
		]
	}{
	\end{tcolorbox}
}

\lstset{
	xleftmargin=0pt,
	framexleftmargin=0pt,
	autogobble=true,
	linewidth=\linewidth,
	breaklines=true,
	columns=fullflexible,
	basicstyle=\small,
}

\begin{document}
	
	\newgeometry{left=0.8in,
		right=0.8in,
		top=0.9in,
		bottom=0.5in}
		
	\begin{multicols}{2}
		\begin{title-box}{moonishblue!70!white}
			{\raggedright \large \textbf{Calculo I: Sustituciones trigonométricas} \par}
			
			\vspace{2mm}
			
			\begin{title-sign}\end{title-sign}
			
			\vspace{5mm}
			
			\emoji{crescent-moon} Una guía para integrar con triángulos. \emoji{crescent-moon}
		\end{title-box}
		
		\vfill
		
		\begin{formula-box}
			{\raggedright \large \textbf{Antes de comenzar} \par}
			
			Al momento de integrar nos encontraremos con sumas y diferencias de cuadrados. En estos casos una posible solución podría ser una sustitución trigonométrica. \par
			\vspace{1mm}
			Para términos de la forma no factorizable:
			\[
			ax^2+bx+c
			\]
			Se pueden completar cuadrados y aplicar la sustitución cuando sea lo más conveniente.
		\end{formula-box}
		
		\begin{formula-box}
			{\raggedright \large \textbf{Sustitución trigonométrica} \par}
			
			Consiste en reescribir los términos de la forma anterior a partir del teorema de Pitágoras:
			\[
			a^2 + b^2 = c^2
			\]
			
			Con esto podremos expresar a la función en términos de un angulo $\theta$:
			\begin{center}
				\begin{multicols}{3}
					\begin{tikzpicture}
						\coordinate (a) at (0,0);
						\coordinate (b) at (1.5,0);
						\coordinate (c) at (1.5,1.5);
						
						\draw (a) -- (b) node[midway, below]{$\sqrt{a^2-x^2}$};
						\draw (b) -- (c) node[midway, right]{x};
						\draw (c) -- (a) node[midway, above]{a};
						
						\pic [draw, <->, angle radius=5mm, angle eccentricity=1.5, "$\theta$"] {angle=b--a--c};
					\end{tikzpicture}
					
					{\footnotesize $x = a\sin{\theta}$}
					
					\columnbreak
					
					\begin{tikzpicture}
						\coordinate (a) at (0,0);
						\coordinate (b) at (1.5,0);
						\coordinate (c) at (1.5,1.5);
						
						\draw (a) -- (b) node[midway, below]{x};
						\draw (b) -- (c) node[ right, xshift=3mm, rotate=-90]{$\sqrt{x^2-a^2}$};
						\draw (c) -- (a) node[midway, above]{a};
						
						\pic [draw, <->, angle radius=5mm, angle eccentricity=1.5, "$\theta$"] {angle=b--a--c};
					\end{tikzpicture}
					
					\vfill
					
					{\footnotesize $x = a\sec{\theta}$}
					
					\columnbreak
					
					\begin{tikzpicture}
						\coordinate (a) at (0,0);
						\coordinate (b) at (1,0);
						\coordinate (c) at (1,1);
						
						\draw (a) -- (b) node[midway, below]{a};
						\draw (b) -- (c) node[midway, right]{x};
						\draw (c) -- (a) node[midway, above, rotate=45]{$\sqrt{a^2+x^2}$};
						
						\pic [draw, <->, angle radius=5mm, angle eccentricity=1.5, "$\theta$"] {angle=b--a--c};
					\end{tikzpicture}
					
					\vfill
					
					{\footnotesize $x = a\tan{\theta}$}
				\end{multicols}
			\end{center}
			
			Cambiando a $x$ y $a$ de lugar se puede trabajar con las otras tres funciones trigonométricas mas.
			
			Al final la mejor sustitución dependerá del problema, sin embargo, comúnmente se calculan $x$ y $dx$ así como la suma o diferencia de cuadrados con respecto a $\theta$.
		\end{formula-box}
		
		\columnbreak
		
		\begin{cut-me-box}
			{\raggedright \large \textbf{"La tabla 11"} \par}
			
			¿Alguna vez te ha pasado que no puedes memorizar todas esas tablas de integrales? Al final resulta que con un poco de sustituciones esto no es necesario.
			
			\begin{center}
				\emoji{fire} ¿Cómo resolverías esta integral? \emoji{fire}
			\end{center}
			
			\[\int \frac{1}{\sqrt{x^2+1}} dx\]
			
			Primero construimos un triangulo rectángulo a partir de la suma de cuadrados que encontramos:
			
			\setlength{\columnsep}{-2cm} % Ajusta solo para este bloque
			
			\begin{multicols}{2}
				
				\begin{tikzpicture}
					\coordinate (a) at (0,0);
					\coordinate (b) at (2,0);
					\coordinate (c) at (2,2);
					
					\draw (a) -- (b) node[midway, below]{1};
					\draw (b) -- (c) node[midway, right]{x};
					\draw (c) -- (a) node[midway, above, rotate=45]{$\sqrt{x^2+1}$};
					
					\pic [draw, <->, angle radius=5mm, angle eccentricity=1.5, "$\theta$"] {angle=b--a--c};
				\end{tikzpicture}
				
				\columnbreak
				
				Se realizan los cálculos con las funciones que más nos convengan:
				\[
				\begin{array}{c}
					\tan \theta = x, \quad
					\sec^2 \theta d\theta = dx \\
					\sec \theta = \sqrt{x^2+1} \\
				\end{array}
				\]
			\end{multicols}
			
			Reemplazamos lo que calculamos en la integral:
			
			\[
			\int \frac{1}{\sec \theta} \sec^2\theta d\theta = 
			\int \sec \theta d\theta
			\]
			
			Resolvemos la integral resultante:
			
			\[
			\int \sec \theta d\theta = 
			\]
			\[
			\int \sec \theta \frac{\sec\theta + \tan\theta}{\tan\theta + \sec\theta} d\theta =
			\int \frac{\sec^2\theta + \sec\theta\tan\theta}{\tan\theta + \sec\theta} d\theta
			\]
			
			¿Notaste algún patrón? Ahora podemos hacer una sustitución simple:
			\[
			\begin{array}{c}
				Sust. \quad u = \tan\theta + \sec\theta \\
				\quad du = \sec^2\theta + \sec\theta\tan\theta d\theta
			\end{array}
			\]
			\[
			\int \frac{du}{u} = \ln| u | = \ln| \tan\theta + \sec\theta | + C
			\]
			Por ultimo deshacemos las sustituciones hasta llegar a términos de $x$:
			\[
			\ln| x + \sqrt{x^2+1} | + C
			\]
		\end{cut-me-box}
	\end{multicols}
	
	
\end{document}
