% !TeX TS-program = lualatex
\documentclass{article}
\usepackage{pgfplots}
\usepackage{csquotes}
\usepackage{fontspec}
\usepackage[a4paper]{geometry}
\usepackage{babel}
\usepackage{multicol}
\usepackage{listings}
\usepackage{lstautogobble}
\usepackage{tikz}
\usepackage[most]{tcolorbox}
\usepackage[colorlinks=true, linkcolor=moonishblue, urlcolor=moonishblue]{hyperref}
\usepackage[dvipsnames]{xcolor}
\usepackage{emoji}
\usepackage{xcolor}
\usepackage{tabularx}

\definecolor{moonishblue}{RGB}{96, 30, 255}

\usetikzlibrary{arrows, positioning, shapes, fit, calc, angles, quotes}

\pgfdeclarelayer{background}
\pgfsetlayers{background,main}




\title{Una práctica ingeniosa para enseñar de todo. Más facil que tu hermana.}

\author{Los tres hijueputas del oriente}

% MapoGoldenPier

\newfontfamily\koreanfont[
Path = ./fonts/, % Adjust this if the folder is different
Extension = .ttf,
UprightFont = MapoGoldenPier
]{}

\newfontfamily\chinesefont[
Path = ./fonts/, % Adjust this if the folder is different
Extension = .ttf,
UprightFont = NotoSerifTC-SemiBold
]{}

\newcommand{\chin}[1]{{\chinesefont #1}}

\newcommand{\kor}[1]{{\koreanfont #1}}

\NewDocumentEnvironment{title-sign}{o}{%
	{\large \textbf{\href{https://www.youtube.com/@Pianitas38}{Alex} - \today}}%
}{}

\newenvironment{formula-box}{
	\begin{tcolorbox}[
		colframe=black, % color del borde
		colback=white, % color del fondo
		arc=1mm, % radio de las esquinas redondeadas
		boxrule=1pt, % grosor del borde
		left=2mm, % padding izquierdo
		right=2mm, % padding derecho
		top=2mm, % padding superior
		bottom=2mm, % padding inferior
		]
	}{
	\end{tcolorbox}
}

\newenvironment{title-box}[1]{
	\begin{tcolorbox}[
		colframe=white, % color del borde
		colback=white, % color del fondo
		arc=1mm, % radio de las esquinas redondeadas
		left=2mm, % padding izquierdo
		right=2mm, % padding derecho
		top=2mm, % padding superior
		bottom=2mm, % padding inferior
		borderline={0.5mm}{0mm}{#1,dashed},
		enhanced,
		boxrule=0.5mm,
		]
	}{
	\end{tcolorbox}
}

\newenvironment{cut-me-box}{
	\begin{tcolorbox}[
		colframe=white, % color del borde
		colback=white, % color del fondo
		arc=1mm, % radio de las esquinas redondeadas
		left=2mm, % padding izquierdo
		right=2mm, % padding derecho
		top=2mm, % padding superior
		bottom=2mm, % padding inferior
		borderline={0.5mm}{0mm}{black!70!white,dashed},
		enhanced,
		boxrule=0.5mm,
		overlay={
			\node[anchor=north, yshift=-1mm, fill=white] at (frame.north east) {\large \emoji{scissors}}; % Coloca el emoji sobre el borde superior
		}
		]
	}{
	\end{tcolorbox}
}

\lstset{
	xleftmargin=0pt,
	framexleftmargin=0pt,
	autogobble=true,
	linewidth=\linewidth,
	breaklines=true,
	columns=fullflexible,
	basicstyle=\small,
}

\begin{document}
	
	\newgeometry{left=0.8in,
		right=0.8in,
		top=0.9in,
		bottom=0.5in}
	
	{\centering \section*{Arquitectura de computadores: Memorias}}
	
	\setcounter{section}{0}
	
	\begin{multicols}{2}
		\subsection{SRAM}
		
		[Static Random Access Memory] En cuanto a capacidad son muy pequeñas, pero son las más rápidas, con el costo de ser las mas grandes y caras.
		
		\subsubsection{Flip flops}
		
		Responden en un solo ciclo de reloj, usadas para los registros de la CPU.
		
		\subsubsection{Cache}
		
		La verdadera SRAM, usada en la cache del procesador, con el costo de responder en 2 ciclos de reloj.
		
		\begin{itemize}
			\item Par leer: Conecta las salidas y lee por las dos lineas.
			\item Para escribir: Alimenta las dos lineas con los Q y $\overline{Q}$ correspondientes
		\end{itemize}
		
		\subsection{DRAM}
		
		[Dynamic Random Access Memory] Es más barata que las anteriores y usada en la memoria principal. Con el costo de funcionar a base de condensadores, pues debe ser refrescada frecuentemente. Esta memoria presenta R/W asíncronos.
		
		\subsection{DDR-SDAM}
		
		[Double Data Rate Sync Dynamic Random Access Memory] Se usa en las memorias principales actuales como una evolución de la DRAM tradicional. Se caracteriza por funcionar de forma sincrónica y con R/W en ambos flancos del reloj.
		
		\subsection{ROM}
		
		Memorias diseñadas para ser de solo lectura y permitir pocas o incluso ninguna re-escritura.
		
		\subsubsection{ROM}
		
		[Read Only Memory] Las primeras en su clase, con la característica de tener datos cableados manualmente. Duraderas pero extremadamente caras.
		
		\subsubsection{PROM}
		
		[Programable Read Only Memory] Una variación de la ROM tradicional que puedes programar en casa, quemando fusibles para escribir los bits. Sin posibilidad de borrar sus datos.
		
		\subsubsection{EPROM}
		
		[Erasable Programable Read Only Memory] También puedes programarla en casa, pero ahora con la función de borrar los datos con luz UV.
		
		\subsubsection{EEPROM}
		
		[Electric Erasable Programable Read Only Memory] Una variación de la EPROM, con un borrado de datos puramente eléctrico.
		
		\subsection{FLASH}
		
		La memoria usada en los discos SSD actuales. Parecida a las EEPROM, aunque hecha a base de una modificación de los MOSFET con una compuerta flotante (como un capacitor).
	\end{multicols}
	
\end{document}
